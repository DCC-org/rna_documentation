\chapter*{Vorwort}

\chapter*{Danksagung}

\newpage

\tableofcontents
\listoffigures
\begingroup
\let\clearpage\relax
\lstlistoflistings{}
\listoftables
\endgroup

\chapter{Einführung}
\label{chap:einfuehrung}
In dem siebten und achten Semester der Fortbildung zum staatlich geprüften
Informatiker am Heinrich\hyp{}Hertz\hyp{}Europakolleg wird in dem Unterricht
für Rechner\hyp{}Netzwerk\hyp{}Architekturen von mehreren Studentengruppen von
zwei bis drei Personen jeweils ein Windows oder Linux Server vollständig
installiert. Schwerpunkt der auf den Servern zu betreibenden Diensten liegt bei
DNS, Fileservern und DHCP. Jede Gruppe soll die Dienste auf eigenen
virtualisierten Instanzen realisieren. Den Studentengruppen stehen dabei Server
von Dell mit der Modell-Version PowerEdge R720xd zur Verfügung. Die Umsetzung
und verwendete Software ist den Studentengruppen selbst überlassen und frei
wählbar.

\chapter{Basisinstallation}
\label{chap:basisinstallation}
Zur Installation der benötigten Dienste und virtuelle Instanzen stand dem
Projektteam ein physischer Root-Server von dem Hersteller DELL zur Verfügung.
Die Hardware des Servers war bereits vollständig zusammengesetzt und zur
Installation eines Betriebssystems bereit. Bevor dieses installiert werden
konnte musste zunächst das RAID\hyp{}Control eingerichtet werden. Dieses
erstellt zu Beginn eine vom Projektteam definierte Festplattenstruktur. Dabei
wurde festgelegt, dass RAID10 für den logischen Zusammenschluss der physischen
Festplatten genutzt wird. Dabei spiegeln sich die 16 genutzten Festplatten
dauerhaft gegenseitig, sodass im Falle eines Ausfalles einer Festplatte, die
darauf gespeicherten Inhalte weiterhin zur Verfügung stehen.

Anschließend konnte das Linux Betriebssystem ArchLinux über einen
USB\hyp{}Stick auf dem Serversystem installiert werden. Um ein Betriebssystem
von einem USB\hyp{}Stick zu installieren muss der verwendete USB\hyp{}Stick
für eine Installation von diesem vorbereitet werden. Das Projektteam nutze
dazu die Software DD. Diese benötigt ein Image Dokument des zu installierendem
Betriebssystem, welches das Projektteam von der Hersteller\hyp{}Webseite von
ArchLinux heruntergeladen hat. Somit konnte während dem Boot\hyp{}Vorgangs des
pysischen Servers angegeben werden, dass sich die zu installierenden Inhalte
auf dem USB\hyp{}Stick befinden.

\chapter{Fazit}

%\appendix

\printglossaries%

\begingroup
\tolerance1000
\emergencystretch0.5em
\printbibliography[heading=bibnumbered]
\endgroup

\chapter{Anhang}

\input{figures.tex}
\FloatBarrier%
\begin{listing}[ht]
  \inputminted[fontsize=\small]{text}{listings/atop.txt}
  \caption{atop ASCII Logausgabe}
  \label{lst:atop}
\end{listing}

\FloatBarrier%
\begin{center}
  \begin{tabularx}{\textwidth}{p{2.5cm} lX}
  \toprule
    Projekt     & URL                                                   \\
  \midrule
    Puppet      & https://tickets.puppetlabs.com/browse/PA-668          \\
    Puppet      & https://tickets.puppetlabs.com/browse/PUP-7383        \\
    Mcollective & https://tickets.puppetlabs.com/browse/MCO-804         \\
    Grafana     & https://github.com/voxpupuli/puppet-grafana/issues/35 \\
  \bottomrule
\end{tabularx}
\captionof{table}{Gemeldete Bugs in Open Source Projekten}
\label{tbl:fossissues}
\end{center}


\chapter{Erklärung}
Hiermit erklären wir, dass wir die Arbeit selbstständig verfasst und keine
anderen als die angegebenen Quellen und Hilfsmittel benutzt haben. Diese Arbeit
wurde keinem anderen Prüfungsausschuss in gleicher oder vergleichbarer Form
vorgelegt.

\vspace{10ex}
{\centering
\renewcommand{\arraystretch}{0.9}
\begin{tabular}{p{0.25\textwidth}p{0.05\textwidth}p{0.25\textwidth}p{0.05\textwidth}p{0.25\textwidth}}
  \dotfill                    & & \dotfill                      & & \dotfill \\
  \centering\footnotesize{Tim Meusel}& & \centering\footnotesize{Marcel Reuter}& & \centering\footnotesize{Nikolai Luis}%
\end{tabular}
}

%%% Local Variables:
%%% mode: latex
%%% TeX-master: "thesis-de"
%%% End:
