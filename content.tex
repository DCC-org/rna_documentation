\chapter*{Danksagung}

\newpage

\tableofcontents
\listoffigures
\begingroup
\let\clearpage\relax
\lstlistoflistings{}
\listoftables
\endgroup

\chapter{Einführung}
\label{chap:einfuehrung}

\label{Vorbereitung}
Bereits zu Beginn des Projektes, wurde festgelegt, welche Server welche
Aufgaben ausführen sollen. Dabei wurde definiert, das zwei Virtuelle Maschine
jeweils einen DomainController (DC01, DC02) installiert haben müssen. Die 
vorhandene Domäne \texttt{mikado.spiel} soll weiterverwendet werden.

Für die Virtualisierung der Maschinen, wurde auf dem HyperVisor mit dem 
Befehl \texttt{pacman -S libvirt} der Virtualisierungsdienst installiert.
Zusätzlich zu diesem musste eine Virtuelle Netzwerkkarte erstellt werden,
damit die virtuellen Maschinen über einen Netzwerkanbindung verfügen.
Auf dem jeweiligen Client, welche die Virtuellen Maschinen verwalten möchte,
muss über \texttt{pacman -S virt-manager} heruntergeladen und installiert
werden.


\label{Samba}
Als Active Domain Controller, wird ein Samba Server auf einem ArchLinux
System installiert. Dies ist eine Open Source Software, welche unter der
\texttt{GNU General Public Licenses} Das entsprechende Paket kann hier mit dem Befehl
\texttt{pacman -S samba} installiert werden. Um den Samba Server installieren
zu können, muss zuneächst geprüft werden, das kein Samba Dienst bereits
installiert und ausgeführt wird. Anschließend muss in der \texttt{hosts} Datei
des Domain Controllers die Einträge für den DC Server für sich selber
vervollständigt werden.

\label {Bind}
Die Software Bind wird für die generelle Namesauflösung innerhalb der Domäne
benötigt. Sie wird in der Regel immer zusammen mit dem ActiveDirectory
verwendet und installiert. Das Package hierzu muss gesondert mit dem oben
stehenden Befehl heruntergeladen werden.
\chapter{Fazit}

%\appendix

\printglossaries%

\begingroup
\tolerance1000
\emergencystretch0.5em
\printbibliography[heading=bibnumbered]
\endgroup

\chapter{Anhang}

\input{figures.tex}
\FloatBarrier%
%\begin{listing}[ht]
%  \inputminted[fontsize=\small]{text}{listings/atop.txt}
%  \caption{atop ASCII Logausgabe}
%  \label{lst:atop}
%\end{listing}

\FloatBarrier%
\begin{center}
  \begin{tabularx}{\textwidth}{p{2.5cm} lX}
  \toprule
    Projekt     & URL                                                   \\
  \midrule
    Puppet      & https://tickets.puppetlabs.com/browse/PA-668          \\
    Puppet      & https://tickets.puppetlabs.com/browse/PUP-7383        \\
    Mcollective & https://tickets.puppetlabs.com/browse/MCO-804         \\
    Grafana     & https://github.com/voxpupuli/puppet-grafana/issues/35 \\
  \bottomrule
\end{tabularx}
\captionof{table}{Gemeldete Bugs in Open Source Projekten}
\label{tbl:fossissues}
\end{center}


\chapter{Erklärung}
Hiermit erklären wir, dass wir die Arbeit selbstständig verfasst und keine
anderen als die angegebenen Quellen und Hilfsmittel benutzt haben. Diese Arbeit
wurde keinem anderen Prüfungsausschuss in gleicher oder vergleichbarer Form
vorgelegt.

\vspace{10ex}
{\centering
\renewcommand{\arraystretch}{0.9}
\begin{tabular}{p{0.25\textwidth}p{0.05\textwidth}p{0.25\textwidth}p{0.05\textwidth}p{0.25\textwidth}}
  \dotfill                    & & \dotfill                      & & \dotfill \\
  \centering\footnotesize{Tim Meusel}& & \centering\footnotesize{Marcel Reuter}& & \centering\footnotesize{Nikolai Luis}%
\end{tabular}
}

%%% Local Variables:
%%% mode: latex
%%% TeX-master: "thesis-de"
%%% End:
